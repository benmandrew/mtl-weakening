\subsection{Weakening on a Counterexample}\label{subsec:cex-relative-weakening}

Model checkers generally produce a specific type of infinite counterexample trace, called a \emph{lasso trace}.

\begin{definition}[Lasso traces]
A trace $\pi$ is \emph{lasso} if it can be separated into a finite prefix $\pipre$ and an infinitely repeating finite suffix $\pisuf$, forming
\begin{equation}
\pi=\pipre(\pisuf)^\omega.
\end{equation}
% \centereqn{\pi=\pipre(\pisuf)^\omega.}
This restricts us to a subset of infinite traces that can be finitely represented. The finite length of a lasso trace is then defined as $|\pi|=|\pipre| + |\pisuf|$.
\end{definition}
%
We show that we can prove properties of an entire infinite lasso trace using only a finite \emph{covering interval}. Without this, we may need to iterate over the entire infinite trace, impacting completeness.
%
\begin{definition}[Covering intervals]
The suffix-covering interval of $\pi$, defined with respect to an interval $[a,b]$, is
\begin{equation}
\covering_\pi([a,b]) = [a, \min(b,\rightidx_\pi(a))]
\end{equation}
% \centereqn{\covering_\pi([a,b]) = [a, \min(b,\rightidx_\pi(a))]}
where
\begin{equation}
\rightidx_\pi(a) =
\begin{cases*}
  |\pi|        & if $a<|\pipre|$ \\
  a+|\pisuf|-1 & otherwise.
\end{cases*}
\end{equation}
\end{definition}

\begin{lemma}[Lasso trace coverage]\label{lemma:lasso-trace-coverage}
    Let $\phi$ be an \MTL{} formula, $\pi$ be a lasso trace, and $a\in\N$. If for all $t\in[a,\rightidx_\pi(a)]$ we have $\sat{\pi}{t}\phi$, then for all $t'\in\N$ with $t'\geq a$ we have $\sat{\pi}{t'}\phi$.
\end{lemma}

\begin{proof}
    We assume that for all $t\in[a,\rightidx_\pi(a)]$ we have $\sat{\pi}{t}\phi$, and, taking an arbitrary $t'\in\N$ with $t'\geq a$ want to prove that $\sat{\pi}{t'}\phi$. There are two cases. Firstly, if $t'<|\pi|$ then we know that this is within the $[a,\rightidx_\pi(a)]$ range and so we have $\sat{\pi}{t'}\phi$. Otherwise, if $t'\geq|\pi|$, we split $\pi$ into its prefix $\pipre$ and infinitely repeating suffix $\pisuf$, and want to prove that $\sat{\pipre(\pisuf)^\omega}{t'}\phi$. As $t'\geq|\pi|$, we can split it into $t'=|\pipre| + n\cdot|\pisuf| + m$ for some $n,m\in\N$ with $n\geq 1$ and $m<|\pisuf|$.
    \begin{align}
    \begin{split}
    & \sat{\pipre(\pisuf)^\omega}{|\pipre| + n\cdot|\pisuf| + m}\phi\\
    \implies& \sat{(\pisuf)^\omega}{n\cdot|\pisuf| + m}\phi\\
    \implies& \sat{(\pisuf)^\omega}{m}\phi\\
    \implies& \sat{\pipre(\pisuf)^\omega}{|\pipre| + m}\phi
    \end{split}
    \end{align}
    We know that $|\pipre| + m$ is in the $[a,\rightidx_\pi(a)]$ interval, so we have $\sat{\pi}{t'}\phi$.
    %
    \qed
\end{proof}

We also define the \emph{optimality} of weakenings, used to prove that \cegiw{} will not produce an interval weakening that is any stronger than it needs to be.

\begin{definition}[Optimality of right-extensions and \mbox{-contractions}]
    An interval $I'$ is an optimal right-extension (resp.\ \mbox{-contraction}) of an interval $I$ with respect to a context $C$, \MTL{} formulae $\psi$ and $\psi'$, a temporal operator $\triangle\in\{\mathcal{U},\mathcal{R}\}$, trace $\pi$, and time-step $t$, if
    \begin{equation}
    \sat{\pi}{t}C[\psi\bintemporal{I'}\psi']
    \end{equation}
    % \centereqn{\sat{\pi}{t}C[\psi\bintemporal{I'}\psi']}
    and either (a) $I=I'$, or (b) there exists no strict right-contraction (resp.\ \mbox{-extension}) $I''$ of $I'$ such that $\sat{\pi}{t}C[\psi\bintemporal{I''}\psi']$.
\end{definition}
%
\begin{algorithm}[t]
\SetAlgoNlRelativeSize{-1}
\caption{Weakening within a context $C$}
\label{alg:aux}

\Function{\weaken($C$, $\psi\bintemporal{\Iorig}\psi'$, $\pi$)}{
    \Return $\weakenaux(C, 0)$
}
\BlankLine

\Function{\weakenaux($C$, $t$)}{
    \uIf{$C = [-]$}{
        \uIf{$\triangle = \mathcal{U}$}{
            \Return \weakenuntildirect$(\psi, \psi', t)$
        }
        \uElse(\tcp*[h]{$\triangle = \mathcal{R}$}){
            \Return \weakenreleasedirect$(\psi, \psi', t)$
        }
    }
    $\cdots$\\
    \uElseIf{$C = C_l \until{I} \phi_r$}{
        \Return \weakenuntilleft$(C_l, \phi_r, I, t)$
    }
    \uElseIf{$C = \phi_l \until{I} C_r$}{
        \Return \weakenuntilright$(\phi_l, C_r, I, t)$
    }
    \uElseIf{$C = C_l \release{I} \phi_r$}{
        \Return \weakenreleaseleft$(C_l, \phi_r, I, t)$
    }
    \uElseIf{$C = \phi_l \release{I} C_r$}{
        \Return \weakenreleaseright$(\phi_l, C_r, I, t)$
    }
}
\end{algorithm}

The proof of correctness and optimality follows the inductive structure of \cegiw{}, with base cases for directly weakening the intervals of $\until{I}$ and $\release{I}$ (\cref{lemma:weakening-algorithm-direct-until,lemma:weakening-algorithm-direct-release}), and inductive cases for weakening within both operators on either side (\cref{lemma:weakening-algorithm-inductive-until-left,lemma:weakening-algorithm-inductive-until-right,lemma:weakening-algorithm-inductive-release-left,lemma:weakening-algorithm-inductive-release-right}). Note that the temporal subformula $\psi\bintemporal{\Iorig}\psi'$ with original interval $\Iorig$ is globally visible in all of the recursive function calls.

\begin{algorithm}[t]
\SetAlgoNlRelativeSize{-1}
\caption{Directly weakening interval of $\mathcal{U}$}
\label{alg:weaken-direct-until}
\Function{\weakenuntildirect($\psi_l$, $\psi_r$, $[a,b]$, $t$)}{
    \For{$i \gets a$ \KwTo $\rightidx_\pi(a)$}{
        \If{$\pi, t+i\vDash \psi_r$}{\label{line:weaken-direct-until-1}
            \Return $[a, \max(b,i)]$
        }
        \If{$\pi, t+i\nvDash \psi_l$}{\label{line:weaken-direct-until-2}
            \textbf{break}
        }
    }
    \Return $None$
}
\end{algorithm}

\begin{lemma}[$\mathcal{U}$ base case]\label{lemma:weakening-algorithm-direct-until}
    Let $I$ be an interval, $\psi_l$ and $\psi_r$ \MTL{} formulae, and $\pi$ a lasso trace. Then, for all timepoints $t\in\N$ with
    \begin{equation}
    I'=\weakenuntildirect(\psi_l,\psi_r,I,t),
    \end{equation}
    % \centereqn{I'=\weakenuntildirect(\psi_l,\psi_r,I,t),}
    \noindent either $I'$ is an optimal right-extension of $I$ such that $\sat{\pi}{t}\psi_l\until{I'}\psi_r$, or $I'=None$, in which case there exists no such interval.
\end{lemma}

\begin{proof}
    We take an arbitrary $t$. Our proof for \cref{alg:weaken-direct-until} uses the loop invariant that $\forall j\in[a,\rightidx_\pi(a)]$ with $j<i$ (where $I=[a,b]$), we have that $\nsat{\pi}{t+j}\psi_r$ and $\sat{\pi}{t+j}\psi_l$. On first entry to the loop there is no such $j$, so this is trivially true. On reaching the end of the loop body, we know that $\nsat{\pi}{t+i}\psi_r$ and $\sat{\pi}{t+i}\psi_l$, and so in combination with the loop invariant we know that $\forall j\in I$ where $j\leq i$, we have $\nsat{\pi}{t+j}\psi_r$ and $\sat{\pi}{t+j}\psi_l$. Thus, the loop invariant is preserved. Suppose at the start of iteration $i$ the loop invariant holds. If $\sat{\pi}{t+j}\psi_r$ on \cref{line:weaken-direct-until-1} then we return $I'=[a,\max(b,i)]$. This is an optimal right-extension of $I$ and we have that $\sat{\pi}{t}\psi_l\until{I'}\psi_r$.
    
    If $None$ is returned, then either we broke out of the loop early because for some $i\in[a,\rightidx_\pi(a)]$ we have $\nsat{\pi}{t+i}\psi_l$ at \cref{line:weaken-direct-until-2}, or we ran the loop to completion. In the first case, we know that $\nsat{\pi}{t+i}\psi_r$ as this is checked before at \cref{line:weaken-direct-until-1}, and so combining with the loop invariant we know that $\psi_r$ never held up until $\psi_l$ stopped holding, and so there is no right-extension $I'$ for which $\sat{\pi}{t}\psi_l\until{I'}\psi_r$.
    
    In the second case, by the loop invariant we have that for all $i\in[a,\rightidx_\pi(a)]$ we have $\sat{\pi}{t+i}\psi_l$ and $\nsat{\pi}{t+i}\psi_r$. By \cref{lemma:lasso-trace-coverage} we then have the same for all $i\in\N$ with $i\geq a$, and so there exists no right-extension $I'$ of $I$ that satisfies $\sat{\pi}{t}\psi_l\until{I'}\psi_r$.
    %
    \qed
\end{proof}


\begin{lemma}[$\mathcal{R}$ base case]\label{lemma:weakening-algorithm-direct-release}
    Let $I$ be an interval, $\psi_l$ and $\psi_r$ \MTL{} formulae, and $\pi$ a lasso trace. Then, for all timepoints $t\in\N$ with
    \begin{equation}
    I'=\weakenreleasedirect(\psi_l,\psi_r,I,t),
    \end{equation}
    % \centereqn{I'=$\weakenreleasedirect$(\psi_l,\psi_r,I,t),}
    \noindent either $I'$ is an optimal right-contraction of $I$ such that $\sat{\pi}{t}\psi_l\release{I'}\psi_r$, or $I'=None$, in which case there exists no such interval.
\end{lemma}

\begin{proof}
Proof in \cref{sec:appendix}.
%
\qed
\end{proof}

We prove the inductive cases with an \MTL{} context $C$, an interval $I$, \MTL{} formulae $\psi$ and $\psi'$, a temporal operator $\triangle\in\{\mathcal{U}, \mathcal{R}\}$, and a lasso trace $\pi$. We use the induction hypothesis $P(C)$, that for all timepoints $t\in\N$ with $I'=\weakenaux(C,\psi\bintemporal{I}\psi',t)$, if $I'$ is an interval then $\pi,t\vDash C[\psi\bintemporal{I}\psi']$, and
\begin{compactitem}
    \item[1.] If $\triangle=\mathcal{U}$, then $I'$ is an optimal right-extension of $I$;

    \item[2.] If $\triangle=\mathcal{R}$, then $I'$ is an optimal right-contraction of $I$.
\end{compactitem}
If $I'=None$, then there exists no such interval in each case.


\begin{algorithm}[t]
\SetAlgoNlRelativeSize{-1}
\caption{Weakening within $\mathcal{U}$ on the left}
\label{alg:aux-until-left}

\Function{\weakenuntilleft($C$, $\phi$, $[a,b]$, $t$)}{
    $\bfin \gets \min(b, \rightidx_\pi(a))$ \\
    $intervals \gets [\;]$ \\
    \For{$i \gets a$ \KwTo $\bfin$}{
        \If{$\pi, t + i \vDash \phi$}{\label{line:weaken-until-left-1}
            \If{$i = a$}{
                \Return $\Iorig$
            }
            \Return interval in $intervals$ with maximal absolute difference to $\Iorig$
        }
        $I \gets \weakenaux(C, t + i)$\\
        \If{$I = None$}{\label{line:weaken-until-left-2}
            \Return $None$
        }
        append $I$ to $intervals$
    }
    \Return $None$
}
\end{algorithm}

\begin{lemma}[$\mathcal{U}$-left inductive case]\label{lemma:weakening-algorithm-inductive-until-left}
    Let $C$ be an \MTL{} context, $I$ and $J$ intervals, $\phi$, $\psi$, and $\psi'$ \MTL{} formulae, $\triangle\in\{\mathcal{U}, \mathcal{R}\}$ a temporal operator, and $\pi$ a lasso trace. If $P(C)$ holds, then so does $P(C\until{J}\phi)$.
\end{lemma}

\begin{proof}
    For \cref{alg:aux-until-left} we assume the inductive hypothesis $P(C)$ and want to prove $P(C\until{J}\phi)$. We take an arbitrary $t$ and distinguish two cases, according to whether $\triangle$ is $\mathcal{U}$ or $\mathcal{R}$. In either case, by the induction hypothesis each recursive call evaluates to either $None$ or an optimal interval $I'$ related to $I$ by the corresponding relation (right-extension or \mbox{-contraction} respectively) such that $\sat{\pi}{t+t'} C[\psi\bintemporal{I'}\psi']$.
    
    We use the loop invariant that, for all $j\in\covering_\pi(J)$ with $j<i$, we have that $\nsat{\pi}{t+j}\phi$ and that $I'=\weakenaux(C,\psi\bintemporal{I}\psi',t+j)$ is an interval such that $\sat{\pi}{t+j} C[\psi\bintemporal{I'}\psi']$. On first entry to the loop there is no such $j$, so this is trivially true. On reaching the end of the loop body, we know that $\weakenaux(C,\psi\bintemporal{I}\psi',t+i)\neq None$ from \cref{line:weaken-until-left-2}, and so by the induction hypothesis the recursive call must have produced a suitable interval $I'$. As we also know that $\nsat{\pi}{t+i}\phi$ from \cref{line:weaken-until-left-1}, the loop invariant is thus preserved for $j\leq i$. Suppose at the start of iteration $i$ the loop invariant holds. If $\weakenaux(C,\psi\bintemporal{I}\psi',t+i) = None$ at \cref{line:weaken-until-left-2} then by the induction hypothesis we know that there is no suitable interval $I''$ for which $\sat{\pi}{t+i} C[\psi\bintemporal{I''}\psi']$, and by the loop invariant that there is no $j<i$ for which $\sat{\pi}{t+j}\phi$. Thus, there is no suitable interval $I''$ for which $\sat{\pi}{t} (C\until{J}\phi)[\psi\bintemporal{I''}\psi']$. If $\sat{\pi}{t+j}\phi$ at \cref{line:weaken-until-left-1} then we split on whether it is our first iteration or not. If $i=a$ (where $J=[a,b]$) then we know that $\sat{\pi}{t+a}\phi$, and so any interval will work. We simply return the original interval $\Iorig$.

    Otherwise, by the loop invariant we know that for all $j\in\covering_\pi(J)$ with $j<i$ --- of which there must be at least one as $i>a$ --- we have an interval $I'$ such that $\sat{\pi}{t+j} C[\psi\bintemporal{I'}\psi']$. Applying \cref{lemma:extension-weakening-order} if $\triangle=\mathcal{U}$, or \cref{lemma:contraction-weakening-order} if $\triangle=\mathcal{R}$, we obtain a maximum interval $I''$ such that for all $j\in\covering_\pi(J)$ with $j<i$ we have $\sat{\pi}{t+j} C[\psi\bintemporal{I''}\psi']$. If $\covering_\pi(J)=J$ then we have
    \begin{equation}
    \sat{\pi}{t} (C\until{J}\phi)[\psi\bintemporal{I''}\psi'].
    \end{equation}
    % \centereqn{\sat{\pi}{t} (C\until{J}\phi)[\psi\bintemporal{I''}\psi'].}
    Otherwise, if $\covering_\pi(J)=[a,\rightidx_\pi(a)]$, then by \cref{lemma:lasso-trace-coverage} for all $k\in\N$ with $k\geq a$ we have $\sat{\pi}{t+k} C[\psi\bintemporal{I''}\psi']$, and so the above holds here too.
    %
    \qed
\end{proof}


\begin{lemma}[$\mathcal{U}$-right inductive case]\label{lemma:weakening-algorithm-inductive-until-right}
    Let $C$ be an \MTL{} context, $I$ and $J$ intervals, $\phi$, $\psi$, and $\psi'$ \MTL{} formulae, $\triangle\in\{\mathcal{U}, \mathcal{R}\}$ a temporal operator, and $\pi$ a lasso trace. If $P(C)$ holds, then so does $P(\phi\until{J}C)$.
\end{lemma}


\begin{lemma}[$\mathcal{R}$-left inductive case]\label{lemma:weakening-algorithm-inductive-release-left}
    Let $C$ be an \MTL{} context, $I$ and $J$ intervals, $\phi$, $\psi$, and $\psi'$ \MTL{} formulae, $\triangle\in\{\mathcal{U}, \mathcal{R}\}$ a temporal operator, and $\pi$ a lasso trace. If $P(C)$ holds, then so does $P(C\release{J}\phi)$.
\end{lemma}


\begin{lemma}[$\mathcal{R}$-right inductive case]\label{lemma:weakening-algorithm-inductive-release-right}
    Let $C$ be an \MTL{} context, $I$ and $J$ intervals, $\phi$, $\psi$, and $\psi'$ \MTL{} formulae, $\triangle\in\{\mathcal{U}, \mathcal{R}\}$ a temporal operator, and $\pi$ a lasso trace. If $P(C)$ holds, then so does $P(\phi\release{J}C)$.
\end{lemma}
%
Proofs for \cref{lemma:weakening-algorithm-inductive-until-right,lemma:weakening-algorithm-inductive-release-left,lemma:weakening-algorithm-inductive-release-right} are located in \cref{sec:appendix}. We use these supporting lemmas to prove the correctness and optimality of \cegiw{}.
%
\begin{theorem}[Correctness for weakening]\label{theorem:weakening-algorithm-correctness}
    Let $C$ be an \MTL{} context, $I$ and $J$ intervals, $\phi$, $\psi$, and $\psi'$ \MTL{} formulae, $\triangle\in\{\mathcal{U}, \mathcal{R}\}$ a temporal operator, $\pi$ a lasso trace, and $t\in\N$ be a timepoint. Let $I'=\weakenaux(C,\psi\bintemporal{I}\psi',t)$. If $I'$ is an interval then $\pi \vDash C[\psi\bintemporal{I'}\psi']$, and
    \begin{compactitem}
        \item[1.] If $\triangle=\mathcal{U}$, then $I'$ is an optimal right-extension of $I$;
        
        \item[2.] If $\triangle=\mathcal{R}$, then $I'$ is an optimal right-contraction of $I$.
    \end{compactitem}
    If $I'=None$, then there exists no such interval in each case.
\end{theorem}

\begin{proof}
    We use the same induction hypothesis $P(C)$ defined for the preceding inductive lemmas. The non-temporal inductive cases are omitted for brevity.
    \begin{description}
    \item \textsc{Base case $[-]$:} By \cref{lemma:weakening-algorithm-direct-until} if $\triangle=\mathcal{U}$, and \cref{lemma:weakening-algorithm-direct-release} if $\triangle=\mathcal{R}$, we have $P([-])$.
    \item \textsc{Inductive case $C\until{J}\phi$:} Assuming $P(C)$, by \cref{lemma:weakening-algorithm-inductive-until-left} we have $P(C\until{J}\phi)$.
    \item \textsc{Inductive case $\phi\until{J}C$:} Assuming $P(C)$, by \cref{lemma:weakening-algorithm-inductive-until-right} we have $P(\phi\until{J}C)$.
    \item \textsc{Inductive case $C\release{J}\phi$:} Assuming $P(C)$, by \cref{lemma:weakening-algorithm-inductive-release-left} we have $P(C\release{J}\phi)$.
    \item \textsc{Inductive case $\phi\release{J}C$:} Assuming $P(C)$, by \cref{lemma:weakening-algorithm-inductive-release-right} we have $P(\phi\release{J}C)$.
    \end{description}
    %
    \qed
\end{proof}
%
The time complexity of \cref{alg:aux} is $O(|\pi|^{\texttt{td}(\phi)})$. where $\pi$ is the counterexample trace and $\texttt{td}(\phi)$ is the \emph{temporal depth} of the \MTL{} formula $\phi$, i.e.\ the maximum number of nested temporal operators along any path in the syntax tree.
