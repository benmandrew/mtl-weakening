\section{Conclusion}\label{sec:conclusion}

We present \cegiw{}, a novel algorithm for weakening intervals in \MTL{} properties of degraded systems, and prove its correctness and optimality. We demonstrate how \cegiw{} can be used during the design phase to understand system limitations under degradation, and explore how the formalised requirements of a number of real-world systems may be weakened against real implementations. This shows the applicability of \cegiw{} in the design of safety-critical systems for understanding the impacts of system degradation.

\textbf{Future work.}
A current limitation is that we only weaken on the right-hand-side of intervals, when both left- and right-bound modifications can produce valid weakenings. Restricting to only right-bound modifications creates a total order over the search space, so there is always a single optimum when multiple choices exist. Expanding to both left- and right-bound modifications creates a partial order over generated intervals, and so choosing between intervals is much less obvious. Future work will also explore other types of weakening, making syntactic changes to formulae beyond intervals.

\textbf{Availability.} The implementation of CEGIW, full proofs, extended pseudocode, and all case study artefacts are available in our public repository\footnote{\url{https://github.com/benmandrew/CEGIW}}. Scripts are provided to reproduce all tables and examples reported in \cref{sec:evaluation}.
