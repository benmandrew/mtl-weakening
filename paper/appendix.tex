\appendix

\section{Remaining Algorithm and Proofs}\label{sec:appendix}

\begin{algorithm}[t]
\SetAlgoNlRelativeSize{-1}
\caption{Directly weakening interval of $\mathcal{R}$}
\label{alg:weaken-direct-release}
\Function{\weakenreleasedirect($\psi_l$, $\psi_r$, $[a,b]$, $t$)}{
    $\bfin \gets \min(b, \rightidx_\pi(a))$ \\
    \For{$i \gets a$ \KwTo $\bfin$}{
        \If{$\pi,t+i\nvDash\psi_r$}{
            \If{$i=a$}{
                \Return $None$
            }
            \Return $[a,i-1]$
        }
        \If{$\pi,t+i\vDash\psi_l$}{
            \Return $[a,b]$
        }
    }
    \Return $[a,b]$
}
\end{algorithm}

Proof of \cref{lemma:weakening-algorithm-direct-release}, directly weakening the interval of $\mathcal{R}$.

\begin{proof}
    We take an arbitrary $t$. Our proof for \cref{alg:weaken-direct-release} uses the loop invariant that for all $j\in\covering_\pi(I)$ with $j<i$, we have that $\sat{\pi}{t+j}\psi_r$ and $\nsat{\pi}{t+j}\psi_l$. On first entry to the loop there is no such $j$, so this is trivially true. On reaching the end of the loop body, we know that $\nsat{\pi}{t+i}\psi_r$ and $\sat{\pi}{t+i}\psi_l$, and so in combination with the loop invariant we know that for all $j\in\covering_\pi(I)$ with $j\leq i$, we have that $\nsat{\pi}{t+j}\psi_r$ and $\sat{\pi}{t+j}\psi_l$. Thus the loop invariant is preserved. Suppose at the start of iteration $i$ the loop invariant holds. If $\nsat{\pi}{t+i}\psi_r$ then we have two cases:
    \begin{enumerate}
        \item If $i=a$ we have that $\nsat{\pi}{t+a}\psi_r$, so for all possible right-contractions $I'$ of $I$ we have that $\nsat{\pi}{t}\psi_l\release{I'}\psi_r$. Thus, there is no suitable interval and we return $None$.
    
        \item Otherwise, we return $I'=[a,i-1]$, which is an optimal right contraction of $I$. By the loop invariant we know for all $j\in\covering_\pi(I)$ with $j<i$ that $\sat{\pi}{t+j}\psi_r$, so we can conclude that $\sat{\pi}{t}\psi_l\release{I'}\psi_r$.
    \end{enumerate}
    If during this iteration $i$ we have that $\sat{\pi}{t+i}\psi_l$, then as this is after the above case is checked for we know that $\sat{\pi}{t+i}\psi_r$, and in combination with the loop invariant we have that $\sat{\pi}{t}\psi_l\release{I}\psi_r$. If we run the loop to completion, then by the loop invariant we know that for all $i\in\covering_\pi(I)$ we have $\sat{\pi}{t+j}\psi_r$. If $\covering_\pi(I)=I$ then we have
    $$
    \sat{\pi}{t}\psi_l\release{I}\psi_r
    $$
    If $\covering_\pi(I)=[a,\rightidx_\pi(a)]$ for some $a\in\N$, then by \cref{lemma:lasso-trace-coverage} we have that for all $i\in\N$ with $i\geq a$ we have $\sat{\pi}{t+j}\psi_r$, and so the above holds here too.
    %
    \qed
\end{proof}

\begin{algorithm}[t]
\label{alg:aux-until-right}
\SetAlgoNlRelativeSize{-1}
\caption{Weakening within $\mathcal{U}$ on the right}
\Function{\weakenuntilright($C$, $\phi$, $[a,b]$, $t$)}{
    $\bfin \gets \min(b, \rightidx_\pi(a))$ \\
    $intervals \gets [\;]$ \\
    \For{$i \gets a$ \KwTo $\bfin$}{
        $I \gets \weakenaux(C, t + i)$\\
        \If{$I \neq None$}{
            append $I$ to $intervals$
        }
        \If{$\pi, t + i \nvDash \phi$}{
            \textbf{break}
        }
    }
    \If{intervals is empty}{
        \Return $None$
    }
    \Return interval in $intervals$ with minimal absolute difference to $\Iorig$
}
\end{algorithm}

Proof of \cref{lemma:weakening-algorithm-inductive-until-right}, weakening within $\mathcal{U}$ on the right.

\begin{proof}
    We assume the inductive hypothesis $P(C)$, and want to prove $P(\phi\until{J}C)$. We take an arbitrary $t$. We distinguish two cases, according to whether $\triangle$ is $\mathcal{U}$ or $\mathcal{R}$. In either case, by the induction hypothesis each recursive call evaluates to either $None$ or an interval $I'$ related to $I$ by the corresponding relation (right-extension or \mbox{-contraction} respectively) such that $\sat{\pi}{t+t'} C[\psi\bintemporal{I'}\psi']$.
    
    We use the loop invariant that for all $j\in\covering_\pi(J)$ with $j<i$, we have that $\sat{\pi}{t+j}\phi$. On first entry to the loop there is no such $j$, so this is trivially true. On reaching the end of the loop body, we know for all $j\in\covering_\pi(J)$ with $j\leq i$ that $\sat{\pi}{t+j}\phi$, and so the loop invariant is preserved. Suppose at the start of iteration $i$ the loop invariant holds. If $\nsat{\pi}{t+i}\phi$ we exit the loop, and if the list of intervals is empty then by the loop invariant we know that there are no appropriate intervals $I'$ for any $j\in\covering_\pi(J)$ with $j<i$ for which $\sat{\pi}{t+j} C[\psi\bintemporal{I'}\psi']$, and so the same holds for $\sat{\pi}{t} (\phi\until{J}C)[\psi\bintemporal{I'}\psi']$.
    
    Otherwise, by the loop invariant we know that for all $j\in \covering_\pi(J)$ with $j<i$ we have $\sat{\pi}{t+j}\phi$ and, as we have passed the check above, that for at least one of these $j$ we have an interval $I'$ such that $\sat{\pi}{t+j} C[\psi\bintemporal{I'}\psi']$. Thus, we have
    $$
    \sat{\pi}{t} (\phi\until{J}C)[\psi\bintemporal{I'}\psi']
    $$
    %
    \qed
\end{proof}

\begin{algorithm}[t]
\label{alg:aux-release-right}
\SetAlgoNlRelativeSize{-1}
\caption{Weakening within $\mathcal{U}$ on the right}
\Function{\weakenreleaseright($C$, $\phi$, $[a,b]$, $t$)}{
    $\bfin \gets \min(b, \rightidx_\pi(a))$ \\
    $intervals \gets [\;]$ \\
    \For{$i \gets a$ \KwTo $\bfin$}{
        $I \gets \weakenaux(C, t + i)$\\
        \If{I $=$ None}{
            \Return $None$
        }
        append $I$ to $intervals$\\
        \If{$\pi, t + i \vDash \phi$}{
            \textbf{break}
        }
    }
    \Return interval in $intervals$ with maximal absolute difference to $\Iorig$
}
\end{algorithm}

Proof of \cref{lemma:weakening-algorithm-inductive-release-right}, weakening within $\mathcal{R}$ on the right.

\begin{proof}
    We assume the inductive hypothesis $P(C)$, and want to prove $P(C\release{J}\phi)$. We take an arbitrary $t$. We distinguish two cases, according to whether $\triangle$ is $\mathcal{U}$ or $\mathcal{R}$. In either case, by the induction hypothesis each recursive call evaluates to either $None$ or an interval $I'$ related to $I$ by the corresponding relation (right-extension or \mbox{-contraction} respectively) such that $\sat{\pi}{t+t'} C[\psi\bintemporal{I'}\psi']$.
    
    We use the loop invariant that for all $j\in\covering_\pi(J)$ with $j<i$, we have that $\sat{\pi}{t+j}\phi$. On first entry to the loop there is no such $j$, so this is trivially true. On reaching the end of the loop body, we know that for all $j\in\covering_\pi(J)$ with $j\leq i$ that $\sat{\pi}{t+j}\phi$, so the loop invariant holds. Suppose at the start of iteration $i$ the loop invariant holds. If $\nsat{\pi}{t+i}\phi$ we exit the loop, and if the list of intervals is empty then we know that there are no appropriate intervals $I'$ for any $j\in\covering_\pi(J)$ with $j<i$ for which $\sat{\pi}{t+j} C[\psi\bintemporal{I'}\psi']$, and so the same holds for $\sat{\pi}{t} (C\release{J}\phi)[\psi\bintemporal{I'}\psi']$.
    
    Otherwise, by the loop invariant we know that for all $j\in\covering_\pi(J)$ with $j<i$ we have $\sat{\pi}{t+j}\phi$, and, as we have passed the check above, that for at least one of these $j$ we have an interval $I'$ such that $\sat{\pi}{t+j} C[\psi\bintemporal{I'}\psi']$. Thus, we have
    $$
    \sat{\pi}{t}(C\release{J}\phi)[\psi\bintemporal{I'}\psi']
    $$
    %
    \qed
\end{proof}

\begin{algorithm}[t]
\label{alg:aux-release-left}
\SetAlgoNlRelativeSize{-1}
\caption{Weakening within $\mathcal{R}$ on the left}
\Function{\weakenreleaseleft($C$, $\phi$, $[a,b]$, $t$)}{
    $\bfin \gets \min(b, \rightidx_\pi(a))$ \\
    $intervals \gets [\;]$ \\
    \For{$i \gets a$ \KwTo $\bfin$}{
        \If{$\pi, t + i \nvDash \phi$}{
            \textbf{break}
        }
        $I \gets \weakenaux(C, t + i)$\\
        \If{$I \neq None$}{
            append $I$ to $intervals$
        }
    }
    \If{intervals is empty}{
        \Return $None$
    }
    \Return interval in $intervals$ with minimal absolute difference to $\Iorig$
}
\end{algorithm}

Proof of \cref{lemma:weakening-algorithm-inductive-release-right}, weakening within $\mathcal{R}$ on the right.

\begin{proof}
    We assume the inductive hypothesis $P(C)$, and want to prove $P(\phi\release{J}C)$. We take an arbitrary $t$. We distinguish two cases, according to whether $\triangle$ is $\mathcal{U}$ or $\mathcal{R}$. In either case, by the induction hypothesis each recursive call evaluates to either $None$ or an interval $I'$ related to $I$ by the corresponding relation (right-extension or \mbox{-contraction} respectively) such that $\sat{\pi}{t+t'} C[\psi\bintemporal{I'}\psi']$.
    
    We use the loop invariant that for all $j\in\covering_\pi(J)$ with $j<i$, we have that $\nsat{\pi}{t+j}\phi$ and that $I'=\weakenaux(C,\psi\bintemporal{I}\psi',t+j)$ is an interval such that $\sat{\pi}{t+j} C[\psi\bintemporal{I'}\psi']$. On first entry to the loop there is no such $j$, so this is trivially true. On reaching the end of the loop body, we know that $\weakenaux(C,\psi\bintemporal{I}\psi',t+i)\neq None$, and so by the induction hypothesis the recursive call must have produced a suitable interval $I'$. As we also know that $\nsat{\pi}{t+i}\phi$, the loop invariant is thus preserved for $j\in\covering_\pi(J)$ with $j\leq i$. Suppose at the start of iteration $i$ the loop invariant holds. If $I'=\weakenaux(C,\psi\bintemporal{I}\psi',t+i) = None$ then by the induction hypothesis we know that there is no suitable interval $I''$ for which $\sat{\pi}{t+i} C[\psi\bintemporal{I''}\psi']$, and by the loop invariant that there is no $j\in\covering_\pi(J)$ with $j<i$ for which $\sat{\pi}{t+j}\phi$. Thus, there is no interval $I''$ for which $\sat{\pi}{t} (C\until{J}\phi)[\psi\bintemporal{I''}\psi']$.
    
    If $\sat{\pi}{t+i}\phi$, then we exit the loop. As this check occurred after the recursive call to $\weakenaux$, we know that we have found at least one suitable interval. By the loop invariant we know that for all $j\in\covering_\pi(J)$ with $j<i$ we have an interval $I'$ such that $\sat{\pi}{t+j} C[\psi\bintemporal{I'}\psi']$. Applying \cref{lemma:extension-weakening-order} if $\triangle=\mathcal{U}$, or \cref{lemma:contraction-weakening-order} if $\triangle=\mathcal{R}$, we obtain a maximum interval $I''$ such that for all $j\in\covering_\pi(J)$ with $j\leq i$ we have $\sat{\pi}{t+j} C[\psi\bintemporal{I''}\psi']$, and so
    $$
    \sat{\pi}{t} (\phi\release{J}C)[\psi\bintemporal{I''}\psi']
    $$
    If we run the loop to completion, then by the loop invariant for each $i\in\covering_\pi(J)$ we have a suitable interval $I'$ where $\sat{\pi}{t+j} C[\psi\bintemporal{I'}\psi']$. If $\covering_\pi(J)=J$ then by the same reasoning as above we have a suitable maximum interval. If $\covering_\pi(J)=[a,\rightidx_\pi(a)]$ for some $a$ then by \cref{lemma:lasso-trace-coverage} this property holds for all $i\in\N$ with $i\geq a$, and so in this case we also have a maximum interval.
    %
    \qed
\end{proof}

