\subsection{Applicability of Interval Weakening to Real-World Requirements (RQ2)}\label{subsec:case-studies}
%
\begin{table}[t]
\caption{Interval-weakenable requirements in \fret{} case studies.}
\label{table:weakenable-reqs-case-studies}
\centering
\begin{tabular}{ |l|r|r| } 
\hline
\textbf{Case study} & \textbf{Total requirements} & \textbf{Weakenable requirements} \\\hline
Mechanical lung ventilator~\cite{farrell2024} & 121 & 57 \\\hline
Autonomous drone~\cite{sheridan2025} & 62 & 19 \\\hline
Lift-plus-cruise aircraft~\cite{pressburger2023} & 49 & 29 \\\hline
Aircraft engine controller~\cite{farrell2022} & 42 & 0 \\\hline
Inspection rover~\cite{bourbouh2021} & 15 & 1 \\\hline
Grasping for debris removal~\cite{farrell2022a} & 20 & 0 \\\hline
Robotic patterns~\cite{vazquez2024} & 36 & 11 \\\hline
LMCP challenges~\cite{mavridou2020} & 74 & 7 \\\hline
\textbf{Total} & \textbf{419} & \textbf{124} \\\hline
\end{tabular}
\end{table}
%
\begin{figure*}[t]
    \centering
    \begin{subfigure}[t]{0.475\textwidth}
        \texttt{\textcolor{fret_condition}{upon ControlLoopStart}
        \textcolor{fret_component}{System}
        \textcolor{fret_shall}{shall}
        \textcolor{fret_timing}{within 12 milliseconds}
        \textcolor{fret_response}{satisfy ControlLoopFinish}}
        \caption{Autonomous drone requirement REQ018 describing the maximum time the control loop can take to complete.}
        \label{fig:example-fret-requirement-REQ018}
    \end{subfigure}%
    \hspace{1em}%
    \begin{subfigure}[t]{0.475\textwidth}
        \texttt{\textcolor{fret_condition}{if powerFailure}
        \textcolor{fret_component}{System}
        \textcolor{fret_shall}{shall}
        \textcolor{fret_timing}{for 120 minutes}
        \textcolor{fret_response}{satisfy !off}}
        \vspace{\baselineskip}
        \caption{Mechanical lung ventilator requirement FUN37 describing how long the system must stay on after power failure.}
        \label{fig:example-fret-requirement-FUN37}
    \end{subfigure}%
    \caption{Example \fretish{} requirements from the case studies. Both are taken from systems that are fully implemented and operational in real-world settings.}
\end{figure*}
%
In this section, we analyse existing requirements from a number of real-world case studies to assess how often interval weakening is applicable, and how weakened timing bounds can be interpreted in their respective domains. These requirements are formalised using the Formal Requirements Elicitation Tool (\fret{})~\cite{giannakopoulou2020a} and are written in \fretish{}, a structured natural language that can be translated to \MTL{}~\cite{giannakopoulou2021}. \fretish{} requirements can have a \texttt{\textcolor{fret_timing}{timing}} field, on which we can use interval weakening to weaken the requirement itself. The number of requirements that can be weakened using interval weakening per case study is shown in \cref{table:weakenable-reqs-case-studies}. As an example, the requirement in \cref{fig:example-fret-requirement-REQ018} from the autonomous drone case study~\cite{sheridan2025} uses the \texttt{\textcolor{fret_timing}{within 12 milliseconds}} timing which specifies that if the condition holds in one state, then the consequent must hold within the next twelve states (assuming that state transitions correspond to a millisecond of time passing). The \MTL{} translation of this timing corresponds to the \MTL{} temporal operator $\Diamond_{[0,12]}$, and so the requirement corresponds to
%
% We analyse the requirement sets from a number of case studies, which are formalised using the Formal Requirements Elicitation Tool (\fret{})~\cite{giannakopoulou2020a}. Requirements are written in \fretish{}, a structured natural language that can be translated to \MTL{}~\cite{giannakopoulou2021}. \fretish{} requirements can have a \texttt{\textcolor{fret_timing}{timing}} field, on which we can use interval weakening to weaken the requirement itself. The number of requirements that can be weakened using interval weakening per case study is shown in \cref{table:weakenable-reqs-case-studies}. As an example, the requirement in \cref{fig:example-fret-requirement-REQ018} from the autonomous drone case study~\cite{sheridan2025} uses the \texttt{\textcolor{fret_timing}{within 12 milliseconds}} timing which specifies that if the condition holds in one state, then the consequent must hold within the next twelve states (assuming that state transitions correspond to a millisecond of time passing). The \MTL{} translation of this timing corresponds to the \MTL{} temporal operator $\Diamond_{[0,12]}$, and so the requirement corresponds to
%
\begin{equation}\label{eqn:example-fret-requirement-REQ018}
\Box(\texttt{\textcolor{fret_condition}{ControlLoopStart}}\rightarrow \eventually{[0,12]}(\texttt{\textcolor{fret_response}{ControlLoopFinish}})).
\end{equation}
%
Under system degradation, for example if the onboard communications network is degraded so that commands take longer to reach control surfaces, we may not be able to guarantee this and so would have to weaken the property by extending the interval, giving more time for the system to run its control loop, with an example weakening in
%
\begin{equation}\label{eqn:example-fret-requirement-REQ018-weakened}
\Box(\texttt{\textcolor{fret_condition}{ControlLoopStart}}\rightarrow \eventually{[0,24]}(\texttt{\textcolor{fret_response}{ControlLoopFinish}})).
\end{equation}
%
An example requirement from the mechanical lung ventilator case study~\cite{farrell2024} is shown in \cref{fig:example-fret-requirement-FUN37}, and as the \fretish{} timing \texttt{\textcolor{fret_timing}{for 120 minutes}} corresponds to the \MTL{} temporal operator $\Box_{[1,120]}$, the corresponding \MTL{} property is
%
\begin{equation}\label{eqn:example-fret-requirement-FUN37}
\Box(\texttt{\textcolor{fret_condition}{powerFailure}} \rightarrow  \generally{[1,120]}(\neg\texttt{\textcolor{fret_response}{off}})).
\end{equation}
%
This is a regulatory requirement~\cite{iso2023} and so, if it does not hold in the degraded system, it is critical to know by exactly how much it is violated. We may only be able to guarantee that the ventilator will stay on for at most 90 minutes after \texttt{\textcolor{fret_condition}{powerFailure}}, producing the weakening
%
\begin{equation}\label{eqn:example-fret-requirement-FUN37-weakened}
\Box(\texttt{\textcolor{fret_condition}{powerFailure}} \rightarrow  \generally{[1,90]}(\neg\texttt{\textcolor{fret_response}{off}})).
\end{equation}
%
Of the 127 interval-weakenable requirements in \cref{table:weakenable-reqs-case-studies}, 116 can be weakened by interval extension as in \cref{eqn:example-fret-requirement-REQ018-weakened}, and 11 by interval contraction as in \cref{eqn:example-fret-requirement-FUN37-weakened}.

Several case studies in \cref{table:weakenable-reqs-case-studies} have few or no requirements that can be weakened with interval weakening. These requirements are typically liveness properties specified with the \texttt{\textcolor{fret_timing}{eventually}} timing, which cannot be weakened further, or safety properties specified with the \texttt{\textcolor{fret_timing}{always}} timing, for which interval weakening would not be appropriate. For example, from the grasping for debris removal case study~\cite{farrell2022a},
%
\begin{equation}\label{eqn:example-fret-requirement-always}
\texttt{\textcolor{fret_component}{SV}
\textcolor{fret_shall}{shall}
\textcolor{fret_timing}{always}
\textcolor{fret_response}{satisfy !collide(SV, TGT)}}.
\end{equation}
%
To answer \textbf{RQ2}, we have shown that interval weakening is applicable to a substantial proportion of existing temporal requirements, and that such weakenings have meaningful interpretations in safety-critical domains. We have also answered \textbf{RQ1} by using \cegiw{} to identify problems in a specification, and then deduce useful timing properties in the fixed system.
